\cleardoublepage

\newrefsection

\chapter{外文翻译}

\sectionnonum{摘要}
人们越来越相信毫米波技术将在30-90 GHz的宽频带范围内成为5G无线网络的一部分。实验测量用于对毫米波信道进行建模,以解决诸如人体阴影或由于移动车辆引起的反射等问题。本文提出了一种新的准确定性( Q-D )方法对毫米波信道进行建模。所提出的信道模型允许对场景特定的几何特性、反射衰减和散射、射线阻塞和移动性效应进行自然描述。这种新的信道建模方法对于进一步的测量活动规划、信道模型描述、系统级仿真和网络接入容量估计至关重要。
\section{介绍}
信道模型如3GPP的空间信道模型( SCM )~\cite{3gpptr}和WINNER~\cite{winnerii}是在广泛的信道传播测量活动的基础上建立的,其频率可达6 GHz。IEEE 802.11 ad信道模型~\cite{ieee80211ad}主要针对60 GHz频段的室内(参数化是在确定性环境的基础上创建的,具有场地特异性)。COST 2100~\cite{cost2100}工作在COST信道模型上,该模型依赖于测量活动和参数的提取。最新的METIS 2020 "基于测量的初始信道模型"~\cite{metis2020}指出,可能没有一种"单一"的信道建模方法可以满足需要。

测量活动的挑战来自于室内孤立房间和大型公共区域以及室外密集的城市超高速率热点、高速率区域和更大区域的传播环境。由于在回程和前程传输中需要考虑点对点和点对多点场景,包括移动性、自回程和中继,因此需要考虑以下链路类型:回程BS - BS、前程BS - BS、接入BS - UE和UE - UE链路。
毫米波信道模型需要应对 LOS 和 NLOS 路径损耗、阴影、空间一致性、环境动态、大型天线阵列、移动性效应、漫反射和镜面反射以及不同的极化。

低于 6 GHz 频段的信号传播已得到相当深入的研究;存在许多准确且现实的建模方法,为在该频段工作的通信系统提供链路级和系统级仿真。

无线通信需要对6GHz以上频段进行彻底研究,因为载波频率增加10倍会导致信号传播特性发生质的变化。选择 60 GHz 频段是因为它正好位于 30-90 GHz 毫米波频率范围的中间,主要结果和结论可以在两个方向上推断。

首先,根据弗里斯方程,毫米波的波长会导致明显更高的传播损耗。为了支持高增益天线,信道模型应考虑 TX 和 RX 处信道射线的空间(角度)坐标,并支持所有天线技术。

其次,正如许多工作~\cite{ieee80211ad}、~\cite{rappaport2013a}、~\cite{rappaport2013b}所证实的那样,60 GHz传播信道具有准光性质。由衍射引起的传播并不显着并且实际上不可行。大部分传输功率通过 LOS 和低阶反射路径在发射机和接收机之间传播。为了建立通信链路,必须使用指向 LOS 路径(如果可用)或反射路径之一的可操纵定向天线。准光传播性质的另一个结果是基于图像的光线追踪可以成为预测通道路径的空间和时间分析的有效手段,并且可以用于辅助通道建模。

第三,应该注意的是,在理想反射的情况下,每个传播路径将仅包括一条光线。然而,正如实验研究~\cite{keusgen2013}、~\cite{maltsev2009}、~\cite{sawada2010}所证明的,由于反射表面的精细结构,每个反射路径实际上由在时间和角度域中彼此紧密间隔的许多光线组成。因此,聚类方法直接适用于 60 GHz 室内和室外系统的信道模型,模型的每个聚类对应于 LOS 或 NLOS 反射路径。

第四,毫米波传播的另一个重要方面是其偏振特性。正如毫米波原型~\cite{maltsev2010}的实验研究所证明的那样,由于天线和通道之间的极化特性不匹配而导致的功率下降可能高达 1020 db。

毫米波段的 3D 信道模型特别需要感兴趣的主要使用模型的传播信道的准确时空特性(基本要求);通过可操纵定向天线进行波束形成,支持 TX 和 RX,对天线技术没有限制;提供天线和信号的极化特性并支持传播信道的非平稳特性。

最先进的移动通信信道模型分别描述路径损耗(PL)和空间信道,通常由集群信道脉冲响应(CIR)和角展度统计数据组成。毫米波信道模型的最新工作也遵循这种方法~\cite{rappaport2013b}、~\cite{khan2011},并对实验数据应用不同的聚类分析技术。然而,这种方法适用于非视距条件,这可能不是毫米波通信系统的主要用例。此外,毫米波信号的 PL 特性导致远距离反射较弱,并且靠近 LOS 路径的射线占主导地位。因此,需要新的方法来表征通道迁移率和多普勒效应,以及专注于这些参数表征的新实验测量。此外,繁忙的室外环境的动态性质会导致部分或全部路径阻塞,应适当模拟。阻塞模型是针对室内情况开发的~\cite{jacob2011},但拥有大量汽车和公共汽车、行人和骑自行车者的丰富室外环境可能需要一些新的方法来进行通道测量和测量结果处理。此类测量是在 MiWEBA 项目的框架内进行的~\cite{miweba2013}。

在本文中,我们提出了非平稳环境中毫米波信道建模的新准确定性方法。该方法基于实验结果和光线追踪模拟。第二部分描述了非平稳环境中特殊的长时间实验测量,并提供了信道功率延迟分布分析的结果,该结果导致了信道射线的新分类。第三部分介绍了准确定性信道建模方法,第四部分演示了该方法在街道峡谷场景的毫米波 3D 信道模型开发中的应用。第五节总结了本文。
\section{非静止室外环境中 60 GHz 频段的实验测量结果}
本节介绍“真实世界”户外环境中 60 GHz 频段的实验测量结果。此次测量活动由位于德国柏林波茨坦大街的弗劳恩霍夫亨利希赫茨研究所 (HHI) 在工作时间进行。这些实验的新颖之处在于对长期非静态环境的调查,其特点是存在相对较大的静态物体以及密集的室外城市区域典型的多个动态(移动)障碍物和反射体。在我们的测量过程中,全向天线在 TX 和 RX 侧均使用,主要用于在固定 TX 和 RX 位置固定距离的许多实验中进行毫米波通道探测。还进行了另一项实验,但使用的是移动的 RX 和静止的 TX。在以前的工作中,传统上使用类似测量的结果来评估传播路径损耗指数和信道功率延迟曲线。在目前的工作中,实验测量得到了重建 3D 测量环境中光线追踪模拟结果的补充。正如我们将在以下部分中演示的那样,这种基于测量和模拟的混合信道研究方法可以全面了解现实世界动态环境中的毫米波传播机制。
\subsection{实验测量描述和结果}
测量活动选择的城市室外通道场景如图1所示。可以看出,该场景是一个典型的街道峡谷,街道两侧都是现代建筑,多条车道被人行道隔开,中等大小的树木和街道设施,例如放置在人行道上的公交车站、自行车停放处和座椅。

TX 天线的高度为 3.5 m,代表添加到图 1 所示现有灯柱的小型蜂窝基站的位置。RX 安装在移动推车上,典型的用户天线高度为 1.5 m。第一阶段,对相距 25 m 的固定 TX 和 RX 位置进行多次长期测量,如图 1 所示。第二阶段,也进行移动测量,其中 RX 移动到沿着人行道匀速行驶,距离固定 TX 每侧 50 m。

本次测量活动中使用的通道探测仪基于自主开发的 FPGA 平台,关键参数如表 1 所示。探测仪的主要输出是每 800 μs 采集的每个测量快照的通道脉冲响应 (CIR)。

在第一阶段,针对 TX 和 RX 对的给定静态位置获得了 62,500 个 CIR 快照,这导致单个观察周期为 50 秒。

图中最高峰的延迟约为 83 ns,对应于彼此相距 25 m 的静态 TX 和 RX 之间的视距 (LOS) 路径。根据测得的 CIR,通过对 50 秒观察时间内的接收功率进行平均来获得信道功率延迟曲线(平均因子 N = 62,500)。图 2 显示了 TX 和 RX 对两个不同位置的平均功率延迟曲线 (APDP)。

图 2 中观察到的功率电平变化可能是由从不同到达角度到达 RX 的一根或多根射线引起的,例如,来自不同的反射表面,且 TX 和 RX 之间的总路径长度相同。由于测量带宽有限(250 MHz),这些多径分量未明确解析并产生快速衰落效应。

在使用移动 RX 的第二阶段中,完成了 40 个测量系列,每个测量系列有 62,500 个快照。根据获得的实验结果,使用因子N = 250的移动平均值计算APDP,见图3。

为了验证全向天线的测量结果并收集额外的数据以进行进一步的信道模型描述,实验结果通过重建的 3D 测量环境中的射线追踪模拟进行了补充,如图 4 所示。

从第二级测量环境的射线追踪仿真结果获得的 APDP 绘制在图 5 中的延迟与观测时间图上。

为了进行比较,使用全向天线测量的 APDP(见图 3)为如图 6 中的类似图所示。

比较这两个图,可以看到实验测量和模拟之间非常匹配。此外,光线追踪使我们能够识别导致 APDP 中特定峰值的反射表面,并对相应的光线进行分类。

\subsection{PDP结构分析}
为了详细分析 PDP 结构和统计行为,对每个实验快照应用峰值检测。使用简单的阈值规则来识别最高峰值,请参见图 7 中的示例。峰值对应于最强射线或多径分量 (MPC),如果它们超出估计噪声水平 10 dB,则可解析。

射线延迟与观察时间的位图如图 8 所示,其中指示了所识别的最强射线的出现/消失。值得注意的是,在每次测量中,LOS 射线到达 83 ns,这表示 TX 和 RX 之间的 25 m 间隔造成的延迟。从图8还可以看出,还存在其他稳定射线。这些光线可能与大型街道物体(例如建筑物墙壁)的反射有关。图8中的一些射线由于功率小和/或阻挡而随机出现和消失。这些光线可以与来自远处物体和动态环境中位置较近的元素的反射相关联。 “射线活动”的百分比可以在射线位图上看到,并且假设阻塞随机过程的遍历特性,时间上的活动百分比可以用作非静止环境中的阻塞概率的估计。图 9 显示了街道峡谷测量场景的射线活动百分比直方图(针对一个近壁 TX 和 RX 位置)。

图 9 允许对射线进行分类,以确定它们的行为和对信号功率的影响。活动百分比高于 80\% 的射线是确定性射线(D 射线),强度强且始终存在,直到被阻挡。 D 射线的阻塞百分比估计约为 2-4\%。活跃度在30-80\%左右的射线是随机的(R射线):是远处静止物体的反射,由于传播距离较弱,更容易被遮挡。活动百分比低于30\%的射线是另一种类型的随机射线:随机移动物体的闪烁反射。这些射线不会被“阻挡”,它们实际上“出现”或“闪烁”(F射线)。

在使用移动全向天线进行移动测量时发现了深度信号功率衰落效应(图 10)。这可以通过两种射线之间的干扰来解释:直接 LOS 射线和地面反射射线,由于其带宽有限,信道探测仪无法明确解析该射线。为了证明这一点,二射线通道模型预测的结果也如图10所示。可以看出,这个简单的二射线近似模型与实验数据非常吻合。

因此,毫米波信道可以用包含大部分信号功率的一小组准确定性射线(D射线)和一组具有预定义参数分布的随机射线(R射线)来精确描述。这种混合方法不需要像光线追踪方法那样详细的场景几何形状,并且比纯统计通道描述更准确。下一节将介绍所提出的准确定性信道建模方法的详细信息。

\section{准确定性通道建模方法}
\subsection{渠道模型的总体结构}
在 802.11ad 信道建模文档模型~\cite{ieee80211ad}中,信道脉冲响应的广义描述由下式给出:

其中$h$是生成的信道脉冲响应;$t$, $\phi_{\text{tx}}$, $\theta_{\text{tx}}$, $\phi_{\text{rx}}$, $\theta_{\text{rx}}$分别是发射器和接收器的时间以及方位角和仰角;$A^{(i)}$和$C^{(i)}$分别是第$i$簇的增益和信道脉冲响应;$\delta(\cdot)$是狄拉克$\delta$函数;$T^{(i)}$, $\Phi_{\text{tx}}^{(i)}$, $\theta_{\text{tx}}^{(i)}$, $\Phi_{\text{rx}}^{(i)}$, $\theta_{\text{rx}}^{(i)}$是第$i$个簇的时间角坐标;$\alpha^{(i, k)}$是第$i$第$k$条射线的幅度-th cluster;$\tau^{(i, k)}$, $\phi_{\text{tx}}^{(i, k)}$, $\theta_{\text{tx}}^{(i, k)}$, $\phi_{\text{rx}}^{(i, k)}$, $\theta_{\text{rx}}^{(i, k)}$是第$k$条射线的相对时间角坐标第$i$个簇。


因此,通道可以明确地建模为一组射线及其参数:射线功率和延迟、偏振矩阵、到达和离开角度。通道模型实现需要定义这些参数。
\subsection[]{准确定性 (Q-D) 信道建模}
为了提供信道传播方面的充分建模,提出了准确定性(Q-D)信道建模方法。在这种方法下,首先针对每个信道传播场景,确定几个最强的传播路径(产生接收到的有用信号功率的大部分的 D 射线),并根据部署的几何形状计算它们上的信号传播,以确定性方式确定基站(BS)和用户设备(UE)的位置。每条射线上传递的信号功率是根据理论公式计算的,考虑了自由空间损耗反射、极化特性和 UE 移动性影响:多普勒频移和用户位移。

这些计算中的一些 D-ray 参数可以被视为随机值(例如,反射系数),甚至被视为随机过程(例如,UE 运动)。应当注意,应当考虑的这种准确定性射线的数量取决于所考虑的场景并且在相应的信道模型中可能不同。例如,在下面考虑的室外街道峡谷场景中,通道主要由 4 个 Dray 确定 - LOS、从最近的墙壁反射的一个以及从地面和最近的墙壁反射的一个。对于开放区域场景,D 射线的数量会较少。

在真实环境中,除了 D 射线之外,还有许多其他反射波从不同方向到达接收器。例如,有汽车、树木、灯柱、长凳、像房子一样的大反射器等。所有这些射线在 Q-D 通道模型中都被视为附加随机射线(R 射线)和闪烁射线(F 射线)。不同类型射线的随机属性(阻塞或出现的概率)以第 II.B 节中描述的方式定义。这些光线被描述为随机簇,具有从可用实验数据或更详细的光线追踪模型中提取的指定统计参数。 Q-D模型信道脉冲响应的时域结构如图11所示。
\subsection{街道峡谷毫米波 3D 通道模型示例}
街道峡谷(室外接入超高速率热点)通道模型代表了典型的城市场景:城市街道,人行道沿着高大的长建筑。灯柱上的 AP 与人手上的 UE 之间的接入链路在此场景中进行建模。模型的参数总结于表二中。

对街道峡谷环境进行光线追踪分析,可以定义最重要的光线,应将其视为 D 光线。除直接 LOS 射线外,还包括地面反射射线、最近的墙壁反射射线和墙壁-地面射线。结果表明,对于所有典型的信号偏振情况,这三个 NLOS 射线包含总 NLOS 射线功率的 90\% 以上。

表II定义了场景几何和反射面参数,可用于对射线的完整描述:路径距离、AoA和AoD可以使用图像方法计算,射线功率可以使用路径损耗表达式获得,菲涅尔方程,以及描述粗糙表面损失的公式。随机射线时间参数可以从全向天线的街道峡谷测量中得出(第 II.A 节),角度分布是从射线追踪分析中获得的。主要参数总结于表三(本通道模型示例中未考虑堵塞/外观效应及相应参数)。

射线簇内的子射线也被建模为具有指数衰减功率延迟分布的泊松过程。参数在表 IV 中指定。以同样的方式将星团内射线添加到D射线和R射线中。

\section{结论}
在本文中,我们描述了 Fraunhofer-HHI 研究所和英特尔公司在 MIWEBA 项目期间进行的毫米波频段测量活动~\cite{miweba2013}。街道峡谷场景测量是使用全向天线进行的,支持光线追踪模拟。对这些数据的分析证明了毫米波小基站的可行性,但同时也显示了传播信道的高时间方差。这种高时间方差是由 UE 本身的移动以及经过的行人和汽车引​​起的。

因此,我们开发了新的准确定性方法来对 60 GHz 的室外信道进行建模。该方法基于将毫米波通道模型脉冲响应表示为一些准确定性强射线(D 射线)和一些相对较弱的随机射线(R 射线、F 射线)。

例如,Q-D 方法已应用于街道峡谷场景的 3D 通道模型开发。模型参数是根据实验测量和射线追踪模拟选择的。 MiWEBA 项目交付成果~\cite{miweba2014} 证明,Q-D 方法允许为许多其他场景(接入、回程和设备到设备链路)开发通道模型。



\begingroup
    \linespreadsingle{}
    \printbibliography[title={外文翻译参考文献}]
\endgroup

